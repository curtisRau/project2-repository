\documentclass{article}
\usepackage[utf8]{inputenc}
\usepackage[left=2cm, right=2cm, top=2cm, bottom=2cm]{geometry}

% --------------------------- Code for the bibliography ---------------------------------------------------------------------------------
\usepackage{natbib}												% For fancy in-text citations
\bibliographystyle{plain}

\usepackage{amsmath}
\begin{document}
	\title{MTH 480 Project 2}
	\author{Curtis Rau, Benjamin Brophy}
	\maketitle

\section{Abstract}

\section{Introduction}

Last time we took the Schr\"{o}dinger equation and transformed it to the radial equation in center of mass frame.

\begin{equation}
	\left[ \left( = \frac{\hbar^2}{m} \frac{1}{r^2} \partial_r r^2 \partial_r + \frac{1}{4} m \omega^2 r^2 + \frac{k e^2}{r} \right) \left( - \frac{\hbar^2}{4m} \frac{1}{R^2} \partial_R R^2 \partial_R + m\omega^2 R^2 + \frac{\hbar l(l+1)}{R^2} \right) \right] \psi (r) \theta (R)  = (E_r + E_R) \psi (r) \theta (R)
\end{equation}

we are only interested in the 

\begin{equation}
\left( = \frac{\hbar^2}{m} \frac{1}{r^2} \partial_r r^2 \partial_r + \frac{1}{4} m \omega^2 r^2 + \frac{k e^2}{r} \right) \psi (r)  = E_r \psi (r)
\end{equation}

we can make the equation dimensionless which gives

\begin{equation}
\left( - \partial_\rho^2 + \rho^2 + \frac{\beta}{\rho} \right) u ( \rho ) = \lambda u(\rho ) \\
\left( - \partial_\rho^2 + V (\rho ) \right) u ( \rho ) = \lambda u(\rho )
\end{equation}

Discretize

\begin{equation}
\begin{aligned}
	u(\rho) \to u(\rho_i) = u_i \\
	\rho \to \rho_i = \rho_0 + i h \\
	h = \frac{b-a}{n+1} \\
	a = 0
\end{aligned}
\end{equation}


the linear algebra equation is

\begin{equation}
\hat{A} = 
\left( \begin{array}{cccc}
	 \frac{2}{h^2} + V_1 &    -\frac{1}{h^2}   &                &                        \\
	-\frac{1}{h^2}       & \frac{2}{h^2} + V_2 & -\frac{1}{h^2} &                        \\
	                     &                     &     \ddots     &                        \\
	                     &                     &                &    -\frac{1}{h^2}      \\
	                     &                     & -\frac{1}{h^2} & \frac{2}{h^2} + V_{N-1}
   \end{array} \right)
\end{equation}

\begin{eqnarray}
\hat{A} x = \lambda x \\
S^T A x = \lambda S^T x \\
S^T A S (S^T x) = \lambda (S^T x)
\end{eqnarray}

Eigenvalues unchanged under orthogonal transforms

Eigenvectors change, but the length is preserved

Dot product between vectors is conserved

\section{Methods}

We performed a Jacobian eigenvalue algorithm to 

\section{Analysis}

- for part B, using N = 200 and rhoMax = 5.0 gives the three lowest eigenvalues up to 3 decimal places.

\subsection{Part B}

\begin{table}
	\centering
	\begin{tabular}{ r | l || c | c | r | c | c}
		Size of &           & Householder & Jacobi & Itterations of & Jacobi & Householder \\
		Matrix  & Tolerance &    Time     &  Time  & Jacobi Method  &   E1   &      E1     \\
		\hline
		 100 & 0.1    & 0.01 &  0.15 &  8684 & 3.00030 & 3.22937 \\
		  "  & 0.01   &  "   &  0.20 & 10667 & 2.99924 &    "    \\
		  "  & 0.001  &  "   &  0.21 & 12076 & 2.99923 &    "    \\
		  "  & 0.0001 &  "   &  0.25 & 13200 &    "    &    "    \\
		 200 & 0.1    & 0.1  &  2.5  & 36836 & 3.00228 & 2.99986 \\
		  "  & 0.01   &  "   &  2.9  & 44012 & 2.99981 &    "    \\
		  "  & 0.001  &  "   &  3.2  & 49387 &    "    &    "    \\
		  "  & 0.0001 &  "   &  3.6  & 53888 &    "    &    "    \\
		 300 & 0.1    & 0.38 & 12.2  & 85521 & 3.00024 & 3.00000 \\
		 500 &        & 2.9  &       &       &         & 3.00003 \\
		 750 &        & 9.7  &       &       &         & 3.00006 \\ 
		1000 &        & 54.6 &       &       &         & 2.99967 \\
		
	\end{tabular}
	\caption{rhoMax = 5.0}
\end{table}


\section{Conclusions}

\section{Works Cited}



\end{document}